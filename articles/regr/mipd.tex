\documentclass[10pt]{sigplanconf}
\usepackage[T1]{fontenc}
\usepackage{amsthm}
\usepackage{amsmath}
\usepackage{amssymb}
\usepackage{graphicx}
\usepackage{epstopdf}
\usepackage{hyperref}
\usepackage{alltt}
\usepackage{listings}
\usepackage{array}
\usepackage{extarrows}
\usepackage{setspace}
\usepackage{tikz}
\usepackage{tikz-qtree}
\usetikzlibrary{calc}
\usetikzlibrary{positioning}
\usepackage{floatrow}
\usepackage{mathtools}
\usepackage{makecell}
\usepackage[noline, boxed, procnumbered, linesnumberedhidden, titlenumbered]{algorithm2e} 
\usepackage{enumitem}
\usepackage[firstpage]{draftwatermark}

\newcommand{\techreport}[1]{} % drop #1 to make it a cut
\newcommand{\cL}{{\cal L}}
\newcommand{\eg}{{\em e.g.}}
\newcommand{\ith}{$i^{th}$}
\newcommand{\cut}[1]{}
\newcommand{\todo}[1]{{\bf\em TODO:} #1}
\newcommand*{\pos}[1]{\ensuremath{\color{blue}\uparrow_{\!#1}}}
\newcommand*{\chpos}[2]{\ensuremath{\underset{\color{blue}\uparrow_{\!#2}}{#1}}}

\cut{
\newcommand*{\underuparrow}[1]{\ensuremath{\underset{\uparrow}{#1}}}
\newcommand*{\underuparrowa}[1]{\ensuremath{\underset{\!\uparrow_{\!a}}{#1}}}
\newcommand*{\underuparrowb}[1]{\ensuremath{\underset{\!\uparrow_{\!b}}{#1}}}
\newcommand*{\underuparrowc}[1]{\ensuremath{\underset{\!\uparrow_{\!c}}{#1}}}
\newcommand*{\underuparrowd}[1]{\ensuremath{\underset{\!\uparrow_{\!d}}{#1}}}
}

\newcommand\vizsp[1][.33em]{%
  \mbox{\kern.06em\vrule height.3ex}%
  \vbox{\hrule width#1}%
  \hbox{\vrule height.3ex}}
    
\newtheorem{definition}{Definition}[section]
\newcounter{featurecounter}

\setlist[enumerate]{itemsep=-1mm}

\addtolength{\columnsep}{-4pt}
%\addtolength{\textwidth}{1pt}

\begin{document}

\special{papersize=8.5in,11in}
\setlength{\pdfpageheight}{\paperheight}
\setlength{\pdfpagewidth}{\paperwidth}

\title{A New, More Accurate Approach For Computing Partial Dependence Plots}

\authorinfo{Terence Parr\vspace{-2mm}}
           {University of San Francisco, USA\vspace{-1mm}}
           {parrt@cs.usfca.edu}
\authorinfo{James Wilson\vspace{-2mm}}
           {University of San Francisco, USA\vspace{-1mm}}
           {jdwilson4@usfca.edu}

\maketitle

\begin{abstract}
pithy abstract here
\end{abstract}

\section{Introduction}

In practice, interpreting a machine learning model is just as important as obtaining an accurate model. Feature importance is one such interpretation tool, which indicates the relative predictive power of each feature.  On the other hand, feature importance does not identify the relationship itself between a feature, $x_i$, and the target variable, $y$. A traditional marginal plot shows the relationship between $x_i$ and $y$ but does not isolate the contribution of $x_i$ to $y$.

Partial dependence plots (PDP) \cite{PDP} 

 \cite{ICE}

%\category{D.2.3}{Software Engineering}{Coding - Pretty printers}

%\keywords Formatting algorithms, pretty-printer

% We recommend abbrvnat bibliography style.

\vfill\eject
\bibliographystyle{abbrvnat}

% The bibliography should be embedded for final submission.

\bibliography{mipd}

\end{document}
